\subsection{Test performed}

The team develops three Python scripts that measure the performance
of the hardware running the algorithm for color detection on a video,
picture, and live camera feed from the webcam. To accomplish this, the team
uses using the following Python Libraries: \verb|psutil| and \verb|time|. \verb|psutil|
stands for \textbf{p}rocess and \textbf{s}ystem \textbf{util}ities. The team utilizes this
library to retrieve information on memory and CPU usage for the virtual
machines and Raspberry Pi 4. The team measures this usage as a percentage of total memory/CPU
for each of the 10 times the team runs the algorithm. The time library is used
for time-related operations. In the implementation, the team uses it to
measure how long the algorithm takes to execute in seconds. For the two
videos tested, the team also uses the library to ensure that the
algorithm is running for at least 30 seconds on the videos. This
decision ensures enough strain on the chips to have
measurable results. The team is also implementing the code in this way to
easily run multiple tests. The 30-second cap helps with testing
since the videos are longer than this duration, it makes testing a
faster process. Also, the execution time is
often over 30 seconds due to the time it takes to execute and process
the code itself. These deviations reveal important information about the
performance of the hardware. All tests are done with machines connected
to power since the laptops where the VMs are running could throttle
under battery power or battery saver modes.

\subsection{Picture}

The team manages to get the picture script to function correctly on the VMs,
however, it was the last of the tests. By this point, the team experiences
problems with the Raspberry Pi 4. When it is plugged into power, and
hooked up to an external display, keyboard, and mouse there is no
video signal whatsoever so all the team sees is a black screen. This issue does
not occur for the video or the webcam tests. The Raspberry Pi 4 is
functioning correctly at that time so the team measures data in full for the
other scripts. Unfortunately for the picture, the team cannot test the
Raspberry Pi 4. The team is not particularly sure about what caused this
sudden issue either; the Pi was fully functioning and no changes were
made to the SD card which has the OS loaded onto it. The team tries removing
the SD card and then plugging it in to get the
bootloader to pop up on the screen but that also proves to fail. It
appears that the microcomputer also does not have a bootloader now, and
even though the SD card is formatted correctly and has Linux installed
on it the team is not able to perform the test for the color detection on a
picture for the Raspberry Pi 4 that is being used.

\subsection{Video 1}

The Raspberry Pi 4 compares very well to all 4 of the VMs in every
category measured. The execution time is consistently near 30.0
seconds and for all 10 iterations averages to about 30.054 seconds.
Which is only the 3rd lowest of all the tested devices. This is
impressive considering the weaker processor of the Raspberry Pi 4. The
11th Gen Intel(R) Core(TM) i7-11800H @ 2.30GHz has the best execution
time on average with only 30.018 seconds. The 11th Gen Intel(R) Core(TM)
i7-1165G7 @ 2.80GHz follows with an average of 30.036 seconds. The
execution time amongst all devices however is relatively consistent and
stable around the 30.0 second mark. While the Raspberry Pi 4 is not the
strongest performer, it still holds its own and puts up respectable
numbers. There does appear to be one outlier in the data: the 13th Gen
Intel(R) Core(TM) i7-13700H 2.90 GHz has one iteration where the
execution time comes in at 30.5 seconds, which is the highest of all the
devices. This appears to be an outlier because it is difficult to
replicate, and may be caused by host machine load. Perhaps a
background process in Windows 11 which is the native OS of the machine
causes a sudden spike. If this iteration is ignored, the average
execution time for this machine drops significantly and becomes the 2nd
fastest machine of the 5 tested here. This is more in line with what the
team expects since this machine has the most capable chip.

\begin{figure}
    \centering
    \includegraphics[width=1\linewidth]{Images/video-execution-time.png}
    \caption{Execution Time for Color Detection Algorithm on a Video}
    % \label{fig:enter-label}
\end{figure}

The CPU usage data for the first video reveals that there are slight
changes in performance between running the algorithm on a virtualization
layer and natively on the hardware itself. The Raspberry Pi 4's 4 Core
ARMv8 Cortex-A72 1.8GHz performs the best of all the devices,
boasting the lowest average CPU usage of 0.76\%. This is followed by
the 13th Gen i7 processor at 1.52\% and lastly, the 11th Gen Intel
i7-1165G7 @ 2.80GHz performs the worst with the highest CPU usage of
14.01 \%. However, it is important to note that there are large spikes
in CPU usage for this processor as well. This issue is replicated, but
the latter half of the tests for this processor are more in line with
all the other devices. It is highly likely that there is a background
process impeding the performance and that it finishes halfway through
the tests since there are such large differences between the CPU Usage
from the beginning to the end. So assuming something impeded the quality
of the data for this device, does it change how well the Raspberry Pi 4
does compared to the VMs? The answer is no. The Raspberry Pi is also
easily outperforming the other 11th Gen Processor used in our tests as
well as the Intel(R) Core(TM) i5-7300U CPU @ 2.60GHz.

\begin{figure}
    \centering
    \includegraphics[width=1\linewidth]{Images/video-cpu-usage.png}
    \caption{CPU Usage for Color Detection Algorithm on a Video}
    % \label{fig:enter-label}
\end{figure}

The Memory Usage data also seems to demonstrate similar conclusions that
are made when looking at the CPU usage. Similarly, the Raspberry Pi 4's
average RAM usage is the lowest at 57.19\% with the 13th Gen i7 machine
following closely at 57.25\%. The 11th Gen Intel i7-1165G7 @ 2.80GHz
performs the worst here as well with the highest spike in ten tests
being at 73.80\%. This is a significant increase, however, the team
speculates the cause of such stark differences being a possible
background process terminating mid-test. This graph further
illustrates that this may be the cause of higher resource usage across
the board for this machine since the latter iterations show a
significant drop to a low of 62.30\%. However, this is still higher
than most machines resulting in worse performance across the board. Most
machines perform worse than the Raspberry Pi 4 but the more powerful
machines are the nearest in terms of memory consumption.

\begin{figure}
    \centering
    \includegraphics[width=1\linewidth]{Images/video-memory-usage.png}
    \caption{Memory Usage for Color Detection Algorithm on a Video}
    % \label{fig:enter-label}
\end{figure}

\subsection{Video 2}

There is a significant difference between the two videos
tested. Video 1 shows solid colors that fill the whole screen and
rapidly change. In contrast, in the second clip, various colors are
coming from different edges of the screen colliding with each other
rapidly. The smokey look of the colors in this video leads them to mix.
The additional colors in the frame mean that the algorithm may have a
harder time differentiating between the color it is identifying and the
others. When looking at the execution time for this video, all machines
perform similarly with the lowest average execution time going to the
13th Gen i7 processor at 30.014, and the lowest being the Intel(R)
Core(TM) i5-7300U CPU @ 2.60GHz at 30.046 The difference between these
is so minuscule that they are within fractions of a second. The
execution time is the most stable metric for both video tests and there
are no significant differences between the Raspberry Pi 4 and the VMs
when running the Color Detection Algorithm. Although the VMs tend to
have higher spikes in execution time than when running natively on the
hardware like with the Raspberry Pi 4, the differences are not great
enough to draw significant conclusions about the differences in
performance. VMs may have spikes because of emulation, the load of the
host PC, and limitations in virtualization technology.

\begin{figure}
    \centering
    \includegraphics[width=1\linewidth]{Images/video2-execution-time.jpg}
    \caption{Execution Time for Color Detection Algorithm on Video 2 }
    % \label{fig:enter-label}
\end{figure}

CPU usage for Video 2 is far more interesting. For simplification
purposes, the team argues that all VMs it tests perform similarly with CPU
usage being consistently less than 2\%. However in stark contrast, the
Raspberry Pi 4's 4 Core ARMv8 Cortex-A72 1.8GHz performs significantly
worse with the highest CPU usage of all the machines reaching a high of
45.70\% and averaging at about 27.53\%. This is a drastic change
observed during testing that indicates that the type of video
chosen for the algorithm to analyze greatly affects the resource usage
for processing on the Raspberry Pi's ARMv8 Cortex-A72 processor. It is
important to acknowledge that the high of over 45\% is likely an outlier
as it only occurs once however the rest of the data is between about
23-30\%. This change highlights the inaccuracies of emulating hardware
and the limitations that virtualization software can have with
accurately representing the hardware that is emulated. The
Raspberry Pi suffers significant performance degradation compared to
the VMs but it likely more accurately showcases the true nature of the
intensity of the algorithm itself and how it would run on similar
hardware. Unfortunately since the team could not get Le Potato with its 4
Core ARMv8 Cortex-A53 1.512GHz processor to load and display its OS
properly, it can only speculate as to what causes such big differences
in performance and if it would be similar on this processor as well.

\begin{figure}
    \centering
    \includegraphics[width=1\linewidth]{Images/video2-cpu-usage.png}
    \caption{CPU Usage as a Percent for Color Detection Algorithm on Video 2}
    % \label{fig:enter-label}
\end{figure}

The memory usage for the Raspberry Pi 4 is also the worst with an
average consumption of 77.08\%. A large change from the 57.19\% average
for video 1. Video 2 proves to be more intensive on the system as a
whole for every metric but execution time, where the Raspberry Pi 4
manages to stay within range of all the VMs. The difference in RAM usage
is large here but not as significantly large as the differences in CPU
usage for this test. The laptop whose performance most closely mirrors
the Pi here is the machine with the 11th Gen Intel(R) Core(TM) i7-1165G7
@ 2.80GHz which manages an average CPU usage of 72\%. All other systems
are approximately between 50-53\% CPU usage. The Raspberry Pi 4 more
accurately depicts the performance on real hardware and all VMs even
manage to have lower RAM usage as they do when running the same
algorithm on Video 1.

\begin{figure}
    \centering
    \includegraphics[width=1\linewidth]{Images/video2-memory-usage.png}
    \caption{Memory Usage as a Percent for Color detection Algorithm on Video 2}
    % \label{fig:enter-label}
\end{figure}

\subsection{Webcam}

Running the color recognition algorithm through a live feed from the devices’ webcams is one of the most interesting parts of this project. While the devices ensure that the program is running, they also must focus on running their webcams throughout the program. This task requires the processors to put extra effort into running this program compared to the other programs. This effort might be visible in the results. This part of the project does not use the Intel(R) Core(TM) i5-7300 CPU 2.60 GHz because the device's webcam couldn't function. With all this in mind, the team will look at the results from this section of the experiment.

From the execution time results, the team can see that most devices stay consistently between 30 to 30.07 seconds of runtime for this program. The only exception to this concept is the 11th Gen Intel(R) Core(TM) i7-1165G7 2.80 GHz. Its execution time varies dramatically through all ten iterations of the program. This processor spends an average of 30.375 seconds per iteration. It spends more time on this program than any other device. It takes a maximum of 30.78 seconds to run this program. The lowest runtime for this program is 30.00 seconds on the Raspberry Pi 4, and the lowest average runtime is 30.018 seconds from the 13th Gen Intel(R) Core (TM) i7-13700H 2.90 GHz. The abnormal fluctuation of the execution time for the 11th Gen Intel(R) Core(TM) i7-1165G7 2.80 GHz can be due to the processes running in the background. The strain from the webcam can also cause these inconsistent run time results. Depending on how much feed the processor decides to process, the program can help take up more or less time for each separate iteration. In the case of this processor, it might have to work more than usual to run the webcam, which, in turn, requires an inconsistent amount of time each run to finish the program. Overall, most of the virtual machines compare well to reflect how the hardware of the Raspberry Pi would be in terms of execution time for this webcam program.

\begin{figure}
    \centering
    \includegraphics[width=1\linewidth]{Images/webcam-execution-time.png}
    \caption{Execution Time for Color Detection Algorithm Using a Webcam}
    % \label{fig:enter-label}
\end{figure}

From the perspective of CPU usage, there seems to be no trend on how much CPU the processors use. The range of CPU usage values obtained seems abnormal as well. The team gets values as low as 0.00\% in some iterations from the Raspberry Pi 4 and the 11th Gen Intel(R) Core(TM) i7-11800H 2.30 GHz. This value could mean that less than 0.005\% of the CPU runs this program in these instances. The team believes that it only shows 0.00\% due to some precision limitations. The highest value is 3.20\% in the first run with the 13th Gen Intel(R) Core(TM) i7-13700H 2.90 GHz. The best average CPU is an average of 0.47\% from the Raspberry Pi 4. On the contrary, the worst average comes from the 11th Gen Intel(R) Core(TM) i7-1165G7 2.80 GHz with a value of 1.73\%. In terms of CPU usage, the Raspberry Pi outperforms the virtual machines. This discrepancy could be because background processes from the processor simulating the Raspberry Pi taking up parts of the CPU. Despite this, there is no processor that can fully replicate the result we have gotten from the Pi very accurately.

\begin{figure}
    \centering
    \includegraphics[width=1\linewidth]{Images/webcam-CPU-usage.png}
    \caption{CPU Usage for Color Detection Algorithm Using a Webcam}
    % \label{fig:enter-label}
\end{figure}

Despite the unpredictable results previously mentioned, the memory usage stays consistent with each iteration. Each processor keeps a consistent array of memory usage values. From the results, the lowest memory-consuming processor for this webcam program is the 11th Gen Intel(R) Core(TM) i7-11800H 2.30 GHz. This processor gets an average of 50.34\% memory usage and produces the lowest memory usage value from this program. The lowest amount of memory used for this program is 50.00\%. The 11th Gen Intel(R) Core(TM) i7-1165G7 2.80 GHz takes the most memory to run the program. It achieves a maximum of 69.40\% memory usage to run the program and an average of 68.43\%. Each processor has different ranges, and they maintain within these ranges. Since the team is not relying on media stored in the devices, the memory should not fluctuate as much as it does with the other experiments. Comparing the processor to the Raspberry Pi, none of the processors can simulate the exact same range of memory usage as the Pi did, but some processors can do better than the Pi in this aspect.

\begin{figure}
    \centering
    \includegraphics[width=1\linewidth]{Images/webcam-memory-usage.png}
    \caption{Memory Usage for Color Detection Algorithm Using a Webcam}
    % \label{fig:enter-label}
\end{figure}