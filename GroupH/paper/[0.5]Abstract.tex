This study evaluates the performance of various hardware platforms, including microcomputers like the Raspberry Pi 4, through the implementation of Computer Vision techniques using OpenCV libraries. Amidst a shortage of Raspberry Pi devices in 2020, alternative microcomputers—Raspberry Pi 3, Le Potato, and VirtualBox—were assessed for their capabilities. VirtualBox served as a baseline for comparison, configured to match Raspberry Pi 4 specifications. The evaluation focused on key metrics: execution time, CPU usage, and memory utilization.

The study involves testing pictures, videos, and live webcam feeds, using Python scripts with psutil and time libraries for performance metrics. Results indicate the Raspberry Pi 4's competitive performance, particularly in video processing, despite its less powerful processor. Virtual machines exhibited CPU and memory usage variations, emphasizing emulation's impact.

Notably, the study faced challenges, such as LePotato video output and OS imaging issues. The Raspberry Pi 3 was also in a continuous reboot state that hindered testing. Video analysis revealed the Raspberry Pi 4's stable execution time but highlighted substantial CPU and memory usage differences in diverse video scenarios. Webcam testing showcased the Pi's efficiency in execution time and CPU usage, offering insights into potential background processes affecting virtual machines.

The study demonstrates that the Raspberry Pi 4 performs comparatively with the tested virtual machines, particularly in video processing. Despite the Le Potato having more RAM than the Raspberry Pi 3, it didn't function properly in our experiments. The research also emphasizes the effects of virtualization, illustrating differences in CPU and memory usage across various platforms.