\subsection{VirtualBox}

VirtualBox serves as cross-platform virtualization software, enabling
users to expand the capabilities of their existing computer by
concurrently running multiple operating systems, including Microsoft
Windows, Mac OS X, Linux, and Oracle Solaris. The team's utilization of
VirtualBox aims to standardize the performance evaluation of computer
vision algorithms by configuring identical Base Memory (set at 2000MB to
emulate the 2 GB memory of the Raspberry Pi 4) and employing the same
quad-core CPU. This approach aligns the technological specifications of
the virtual machines as closely as possible with those of the Raspberry
Pi 4.

Additionally, the team ensures uniformity in the operating environment by
employing the same Linux version, specifically Ubuntu 22.04.3, thereby
eliminating any potential impact of the operating system version on the
performance of the color detection algorithm. Four virtual machines,
each configured with these specified parameters, execute the color
detection algorithm to assess metrics such as execution time, CPU usage,
and memory utilization.

It\textquotesingle s noteworthy that each computer, including its
virtual machine, comprises distinct internal components. The primary
objective is to discern whether these variations in internal components
influence algorithm performance and, if so, identify the root causes of
any observed data discrepancies. The color detection algorithm is
executed on each machine to analyze color in real-time through a webcam
feed, identify colors within images, and ascertain color in video
content.

\subsection{Microcomputers}

The team's objective is to thoroughly examine three distinct Reduced Instruction Set Computers (RISC), renowned for their implementation of a streamlined and optimized set of instructions, differing from the specialized instruction sets present in other architectures. Our analysis centers on the Raspberry Pi 4 Model B, the Raspberry Pi 3, and the Le Potato, all of which operate on 64-bit ARM microcomputer architectures. The Raspberry Pi 4 Model B is equipped with the Broadcom BCM2711 System-on-Chip (SoC), featuring a quad-core Cortex-A72 CPU clocked at 1.5 GHz. Throughout our investigation, the Raspberry Pi 4 is configured with 2GB of RAM, enhancing its computational capabilities. Concurrently, the Raspberry Pi 3 and the Le Potato adopt quad-core ARM Cortex-A53 architectures. The Raspberry Pi 3, powered by the Broadcom BCM2837, is furnished with 1GB of RAM. In contrast, the Le Potato by Libre Computer showcases a Mali-450 MP3 GPU in tandem with 2GB of RAM, contributing to its computational prowess and graphics processing capabilities.

Extra things needed to set up a desktop with the

Raspberry Pi 4:

\begin{itemize}
\item
  Micro SD card
\item
  Mouse
\item
  Keyboard
\item
  Desktop
\item
  HDMI to mini HDMI
\item
  Type C Power Cable
\end{itemize}

Raspberry Pi 3:

\begin{itemize}
\item
  Micro SD card
\item
  Mouse
\item
  Keyboard
\item
  Desktop
\item
  HDMI to HDMI
\item
  Type micro USB power Cable
\end{itemize}

Le Potato:

\begin{itemize}
\item
  Micro SD card
\item
  Mouse
\item
  Keyboard
\item
  Desktop
\item
  HDMI to HDMI
\item
  Type micro USB power Cable
\end{itemize}

\subsection{SD cards}

\begin{figure}
    \centering
    \includegraphics[width=1\linewidth]{Images/sd-tbl.png}
    % \caption{Enter Caption}
    % \label{fig:enter-label}
\end{figure}

SD cards play a crucial role as the primary storage medium for
microcomputers like the Raspberry Pi, encompassing the storage of
essential components such as the operating system, software
applications, and user data. Unlike traditional computers, the Raspberry
Pi lacks built-in storage, making an SD card not just a storage solution
but a fundamental component necessary for the proper functionality of
the system.

During the startup process of the Raspberry Pi, the device boots
directly from the SD card. When the user powers up the Raspberry Pi, the
system retrieves essential files from the SD card to initiate the boot
sequence. This includes the bootloader, a small piece of software
responsible for launching the operating system. Both the bootloader and
the operating system itself are typically stored on the SD card.

The choice of an SD card for the Raspberry Pi is a thoughtful process.
It is essential to choose a reliable and appropriately sized SD card to
ensure the smooth and efficient operation of the system. Take, for
example, the SanDisk 32 GB ImageMate microSDXC UHS-1 Memory Card, with
specifications such as Class 10 (C10) for minimum data transfer speeds,
UHS Speed Class 1 (U1) for higher data rates, and compatibility for Full
HD video recording. This microSD card boasts transfer speeds of up to
120MB/s, making it well-suited for applications where rapid data
throughput is crucial.

The read and write speeds of the SD card directly impact the performance
of the Raspberry Pi, especially in scenarios where data access speed is
a critical factor. Therefore, the careful selection of an SD card with
favorable read and write speeds is an important consideration for users
aiming to achieve optimal performance in their Raspberry Pi-based
projects.