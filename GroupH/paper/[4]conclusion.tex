There seems to be no direct substitute for the Raspberry Pi. None of the virtual machines can guarantee similar values to the Raspberry Pi on all tests. If needing to simulate a Raspberry Pi 4's performance on a color recognition algorithm, a processor that will operate closely to the Pi based on how one plans to run the algorithm has to be chosen. To simulate a Raspberry Pi when detecting color in a simple video, use the 11th Gen Intel(R) Core(TM) i7-11800H 2.30 GHz. This processor would also be good to get realistic values of a Raspberry Pi 4's CPU usage and execution time, except for the CPU usage when using the algorithm for a complex video. To accurately simulate the color-detecting algorithm running on a Raspberry Pi 4's memory with a more complicated video or webcam, there is no processor that can perform this task. No processor is suitable for measuring CPU usage during complex video playback. Overall, to use simulators to measure the performance of a Raspberry Pi 4, one must use processors that would best reflect the aspect of the hardware to measure.
